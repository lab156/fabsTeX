\section{Background}

This article assumes basic facts about $K$-algebras (such as tensor
products, ideals, radical ideals),
topological spaces (connectedness), and category theory.

Building on those foundations, the article gives a complete
specification of all finite simple groups.
The definition of a finite simple group of Lie type appears in
Definition \ref{def:fsg-Lie}.
Unexplained notation from this section will be precisely
defined later.

\begin{theorem}\label{thm:classify}
  Every finite simple group is isomorphic to 
\begin{enumerate}
\item a cyclic group of prime order,
\item an alternating group $\Alt_n$ on $n$ letters for some $n\ge 5$,
\item a finite simple group of Lie type, or
\item one of the 26 sporadic groups.
\end{enumerate}
Every group these four families is a finite
simple group.
\end{theorem}

  Finite simple groups of Lie type are classified by certain data of
  the form $(D_r,\rho,p,e)$ (written as ${}^{\rho}D_r(p^e)$), where
  $D_r$ is a connected Dynkin diagram with $r$ nodes, $\rho$ is an
  arrow-forgetful isomorphism of the Dynkin diagram, $p$ is a prime
  number, and $e \in \ring{Q}$ is an exponent.  The explicit list of
  such tuples appears in Definition ~\ref{def:st}.


\begin{theorem}\label{thm:duplicate}
  If two finite simple groups on the
  list of Theorem~\ref{thm:classify} are isomorphic, then that
  isomorphism is one of the following duplicates.
  \begin{itemize}
  \item $\Alt_5 \simeq A_1(2^2) \simeq A_1(5)$.
  \item $\Alt_6 \simeq A_1(3^2)$.
  \item $\Alt_8 \simeq A_3(2)$.
  \item $A_1(7) \simeq A_2(2)$.
  \item $B_3(3) \simeq {}^2A_3(2)$.
  \item $B_r(2^m) \simeq C_r(2^m)$.
  \item Any two finite simple groups of Lie type with
    isomorphic classifying data $(D,\rho,p,e)$ are isomorphic.
  \end{itemize}
\end{theorem}

See \cite{wilson2009finite}.  

\section{Affine varieties}


The material on finite groups of Lie type follows Carter
\cite{carter1985finite} and Humphreys \cite{humphreys2012linear}.

If $I$ is an ideal of a ring $R$, the radical $\sqrt{I}$ of $I$ is the
set of all $f \in R$ such that $f^e \in I$ for some $e\ge 1$.  The
radical of $I$ is an ideal.

Let $K$ be an algebraically closed field.  Let $A_K$ be the set of
finitely generated $K$-algebras without nilpotent elements.  This can
be made into a category, where the morphisms are morphisms of
$K$-algebras, but this category will not be directly needed.  An
affine variety $X$ over $K$ is uniquely determined by an element of
$A_K$.

An algebra $R\in A_K$ determines a set (the spectrum of $R$)
\[
\op{spec}(R) = \op{Hom}_{K-alg}(R,K)
\]
of $K$-algebra homomorphisms from $R$ to $K$.  The spectrum carries a
topology (the Zariski topology) uniquely determined by the condition
that a set $Y$ is closed iff there exists a a subset $S$ of $R$ such
that $Y=Z(S)$, where $Z(S)$ is the set of homomorphisms that vanish on
$S$.  We have $Z(S) = Z(\sqrt{(S)})$, where $(S)$ is the ideal
generated by $S$.

Although the data giving an affine variety $X$ and the algebra $R$ are
one and the same, we make a conceptual distinction, thinking of $X$ as
a geometric object, which has a spectrum (written $X(K)$) and a
coordinate ring (written $K[X]=R$).  We also call $X(K)$ the set of
$K$-points of $X$.

If $Z$ is a closed subset of an affine variety $X$, then the subset
$I(Z)$ on which every homomorphism in $Z$ vanishes is a radical ideal.
The Hilbert Nullstellensatz asserts that there is a bijection between
closed subsets of $X$ and radical ideals of $K[X]$ given by
\[
Z \mapsto I(Z),\quad I\mapsto Z(I).
\]

If $Z$ is a closed subset of an affine variety $X$, then $K[X]/I(Z)$
is an algebra in $A_K$, and $Z$ can be canonically identified with the
spectrum of $K[X]/I(Z)$ via the composition of $K$-algebra
homomorphisms
\[
K[X] \to K[X]/I(Z) \to K.
\]
In this way, each closed subset $Z$ of $X$ is an affine variety with
coordinate ring $K[Z] = K[X]/I(Z)$.

If $R,R'\in A_K$, then $R\otimes R'\in A_K$.  We have a product on
affine varieties given by
\[
K[X \times Y] = K[X]\otimes K[Y].
\]
This product satisfies the universal property of a product.  We warn
that Zariski topology on the product is not the usual product topology
in the theory of topological spaces.

A morphism $f:X \to Y$ of affine varieties is defined to be a
$K$-algebra homomorphism $f^*:K[Y]\to K[X]$ (called the pull-back of
coordinate functions).  A morphism determines a map
\[
f_*: X(K) \to Y(K),
\]
by composition
\[
K[Y] \to K[X] \to K.
\]
This map is continuous.  The data giving $f$ and $f^*$ are one and the
same, but $f^*$ is contravariant, and we make a conceptual distinction
between them.

A point $1$ is an affine variety with $K[1]=K$.  It is a terminal
object; for each affine variety there is a unique morphism $f:X\to 1$.
Specifically, $f^*:K\to K[X]$ is the structure map of the $K$-algebra
$K[X]$.

%An affine variety $X$ is said to be irreducible if $K[X]$ is an
%integral domain.  (Equivalently, $X$ is irreducible if the spectrum
%$X(K)$ cannot be written as the union of two proper closed subsets).
%If $X$ is irreducible, the quotient field $K(X)$ of $K[X]$ is called
%the function field of $X$.  In this case, $K(X)$ has finite
%transcendence degree over $K$, and we say that the dimension of $X$ is
%that degree.

\section{Affine algebraic groups}

Henceforth we refer to a group in the usual sense as abstract groups
to distinguish them from {\it affine groups}, which are group objects
in the category of affine categories, which are described in this
section.

We fix an algebraically closed field $K$.  The objects from the
categories of affine varieties and affine groups are all over this
field $K$.

An affine group over $K$ is defined as an affine variety
$G$, together with morphisms $\mu:G \times G \to G$ (called
multiplication) $i:G \to G$ (called inverse), and $e:1\to G$ (identity
element) such that $\mu$ and $i$ satisfy the usual axioms of a
group. More precisely, we mean the axioms of a group in the category
of affine algebraic groups.  For example, the right inverse property
is that this diagram commutes:
\[
\]
The other axioms are obtained by a similar translation of the group
axioms into categorical language.  The axioms are written explicitly
in the Wiki article on group object \cite{group-object}.  (These
axioms can also be expressed directly in $A_K$ as the axioms of a Hopf
algebra.)

If $G$ is an affine group, then the corresponding operations
on the spectrum
\[
\mu_*:G(K)\times G(K)\to G(K),\quad e_*(1)\in G(K),\quad i_*:G(K)\to G(K)
\]
make $G(K)$ into a group.


A morphism $f:G\to G'$ of affine algebraic groups is a morphism
of affine varieties such that $f(x y) = f(x) f(y)$.  That is,
this diagram commutes:
\[
\]
Unless stated otherwise, a morphism will mean a group homomorphism of
affine algebraic groups.

The point $1$ has the structure of an affine group.

If $G$ and $G'$ are affine groups, then $G \times G'$ also
carries the structure of an affine group in a natural way.

A closed subgroup $H$ of an affine group $G$ is a closed subvariety
that contains the neutral element and such that the inverse and
multiplication on $G$ restrict to $H$.  Then $H$ has the structure of
an affine group.  In what follows, being {\it closed} refers to the
Zariski topology, but never to the algebraic sense in which a binary
operation can be closed.

The kernel of a morphism $\psi:G\to H$ is the closed subset given by
the preimage of $1\in H$.  The kernel of a morphism $\psi:G\to H$ is a
closed subgroup of $G$.  We define a normal subgroup of $G$ to be any
closed subgroup obtained as a kernel of a morphism.

An abelian group is an affine group such that
\[
G \times G \to G \times G \to^\mu G
\]
coincides with $\mu$, where the first morphism swaps the factors.

A solvable group $G$ is defined inductively as an affine group that is
abelian, or as an affine group $G$ that admits a morphism $\psi:G\to
H$, where both $H$ and the kernel of $\psi$ are solvable.

An affine group $G$ is connected, if it is connected as a topological
space.
A Borel subgroup of $G$ is a maximal closed connected solvable
subgroup of $G$.

Let $G$ be an affine group.  There exists an abelian affine group $A$
(the abelianization of $G$) and morphism $G\to A$ with the following
universal property: if $A'$ is any affine abelian group, and morphism
$\psi: G \to A'$, there exists a unique $A\to A'$ such that $\psi$ is
equal to $G\to A \to A'$.  The kernel of $\phi:G\to A$ is the
{\it closed derived} subgroup of $G$.   The kernel does not depend
on the choice of $(A,\phi)$.

%Let $G$ be an affine group.  The set of kernels of all morphisms
%$\psi:G\to A$ with abelian target groups $A$ is partially ordered by
%inclusion. This set of kernels has a unique maximal element $G'$,
%called the {\it closed derived} subgroup of $G$.

Let $G$ be an affine group and let $H$ be a subgroup of $G$.  A closed
subgroup $C$ of $G$ {\it centralizes} $H$ if the morphism of varieties
\[
C \times H \to G, \quad (c,h)\to c h c^{-1}
\]
coincides with the inclusion $H\to G$.  Among all closed subgroups $C$
centralizing $G$, there exists a unique maximal element (under the
ordering by inclusion of closed subsets of $G$).  This is the
centralizer $C_G(H)$ of $H$ in $G$.  The center $Z = Z_G$ of $G$ is
the centralizer in $G$ of $G$:
\[
Z = C_G(G).
\]

We say that a closed subgroup $N$ of $G$ normalizes another closed
subgroup $H$ if the image of the morphism of varieties
\[
N \times H \to G,\quad (n,h) \mapsto n h n^{-1}
\]
lies in $H$.  When $N$ normalizes $H$, there is an action of
$N$ on $H$ given by $(n,h) \mapsto n h n^{-1}$.


\section{The general linear group}

The most important example of an affine algebraic group for us is
$GL(n)$ over $K$.  Let $R=K[x_0,x_{ij}:i=1,\ldots,n;j=1,\ldots,n]$ be
a polynomial ring in $n^2+1$ variables $x_0$ and $x_{ij}$.  Let
$\det(x_{ij})\in R$ be the determinant in $x_{ij}$.  Define $GL(n)$ as
an affine variety by its coordinate ring: $K[GL(n)] =
R/(x_0\det(x_{ij}) - 1)$.  We set $y_{ij} = x_{ij}\otimes 1$ and
$z_{ij} = 1\otimes x_{ij}$ in $K[GL(n)\times GL(n)]$.  Then
\[
K[GL(n)]\otimes K[GL(n)] \simeq
K[y_0,y_{ij},z_0,z_{ij}]/(y_0 \det(y_{ij})-1,z_0\det(z_{ij})-1).
\]

There exists a unique affine algebraic group $GL(n)$ with coordinate
ring $K[GL(n)]$ and
\begin{align*}
  e^*(x_{ij}) &= \delta_{ij} \text{ (Dirac delta function)},&e^*(x_0) &= 1\\
  \mu^*(x_{ij}) &= \sum_{i=1}^n {y_{ik} z_{kj}},&\mu^*(x_0) &= y_0 z_0\\
  i^* &= \text{adjugate formula for matrix inverse},&i^*(x_0)&=\det(x_{ij}).
\end{align*}
The adjugate formula appears here \cite{WA}.

If $a \in GL(n)(K) = GL(n,K) = \op{Hom}_{K-alg}(K[GL(n)],K)$, we write
$a_{ij}$ for $a(x_{ij})\in K$, and write it as an $n\times n$ matrix
with coefficients in $K$.  If $a_0 = a(x_0)$, then
$a_0\det(a_{ij})=1$, so that the determinant of the matrix is nonzero.
In fact, the element $x_0$ appears exactly for the purpose of
expressing the non-zero determinant condition as a polynomial
equation.


\subsection{tori}


When $n=1$, we also write $\Gm = GL(1)$.
The coordinate ring is $K[t,u]/(t u - 1)\simeq K[t,t^{-1}]$.
Also, $\Gm^r
=\Gm \times \cdots \times \Gm$ for the $r$-fold product.
The coordinate ring is
\[
K[\Gm^r]=K[\Gm]\otimes\cdots\otimes K[\Gm]
\simeq K[t_1,t_1^{-1},\ldots,t_r,t_r^{-1}].
\]


A closed subgroup $T$ of $G$ is a {\it maximal torus} if it is a
subgroup isomorphic to $\Gm^r$, with $r\in\ring{N}$ maximal.  We call
$r$ the rank of $G$.  A maximal torus exists in every affine group.


Let $T$ be a maximal torus of $G$.  The character group of $T$
is the set
\[
X^*(T) = \op{Hom}(T,\Gm).
\]
It has the structure of (an abstract) abelian group as follows
by multiplication in the target:
if $\lambda_1,\lambda_2\in X^*(T)$, then $(\lambda_1\lambda_2)(t) =
\lambda_1(t)\lambda_2(t)$, for all $t\in T$.
%$T$
%is isomorphic to $\Gm^r$ for some $r$ and the abelian group
%structure comes by transport of structure along that isomorphism.
%\begin{align*}
%\ring{Z}^r &\simeq
%\op{Hom}_{K-alg}(K[t,t^{-1}],K[t_1,t_1^{-1},\ldots,t_r,t_r^{-1}])
%= \op{Hom}(\Gm^r,\Gm),\\
%(m_1,\ldots,m_r)&\mapsto (t\mapsto t_1^{m_1}\cdots t_r^{m_r}).
%\end{align*}
%In particular,
$X^*(T)$ is a free abelian group of rank $r$,
when $T$ is isomorphic to $\Gm^r$.

The cocharacter group of $T$ is
\[
X_*(T) = \op{Hom}(\Gm,T),
\]
It too has the structure of an abstract abelian group
with multiplication in the target.
%which also has the structure of a free abelian group of rank $r$
%by transport of structure:
%\begin{align*}
%\ring{Z}^r &\simeq
%\op{Hom}_{K-alg}(K[t_1,t_1^{-1},\ldots,t_r,t_r^{-1}],K[t,t^{-1}])
%= \op{Hom}(\Gm,\Gm^r),\\
%(m_1,\ldots,m_r)&\mapsto (t_i\mapsto t^{m_i}).
%\end{align*}
$X_*(T)$ is a free abelian group of rank $r$.

There is a nondegenerate bilinear pairing
\[
\langle\cdot,\cdot\rangle:X^*(T)\times X_*(T)\to \ring{Z},
\]
given by the composition
\[
\langle \lambda,\mu\rangle =
\lambda\circ\mu \in \op{Hom}(\Gm,\Gm) = \ring{Z}
\]


%An affine group $U$ is unipotent if it is isomorphic to a closed
%subgroup of the strictly upper triangular subgroup of $GL(n)$ for some
%$n$.  (This is the closed subgroup of $GL(n)$ given by $x_{ij}=0$ for
%all $i>j$, and $x_{ii}=x_0 = 1$.)

%An example of a unipotent subgroup is the additive group $\Ga$.

\subsection{additive group}

The additive group $\Ga$ is the affine group whose coordinate ring is
a polynomial ring in one variable $K[\Ga] = K[x]$.  The group
structure is uniquely determined by the conditions:
\begin{align*}
  e^*(x)=0,\quad i^*(x)= -x,\quad
  \mu^*(x) = x\otimes 1 + 1\otimes x\in K[x]\otimes K[x].
\end{align*}
%$\Ga$ is isomorphic to the group $U$ of strictly upper
%triangular matrices of $GL(2)$ through
%\[
%K[U]= K[x_0,x_{ij}]/(x_0=1,x_{11}=x_{22}=1,x_{21}=0)
%\simeq K[x_{12}] \simeq K[x].
%\]
The group $\Gm$ acts as automorphisms of $\Ga$ by the morphism
\[
\Gm\times \Ga\to \Ga,\quad K[x]\to K[t,t^{-1}]\otimes K[x],
\quad x\mapsto t\otimes x.
\]

\section{structure theory for almost simple groups}

An affine group is $G$ is {\it almost simple} if it has no proper normal
closed connected subgroup.  An almost simple group is connected
(because the connected component of the identity of any affine group
is a normal closed connected subgroup).

We fix the context in the section.  Let $G$ be almost simple, let $B$
be a Borel subgroup, let $U = B'$ be its closed derived subgroup, and
let $T$ be a maximal torus of $G$ that is a subgroup of $B$.

\subsection{positive roots}

Let $f:X\to \Ga$ be an isomorphism from a
closed subgroup $X$ of $U$ to $\Ga$.  Assume that $X$ is normalized by
$T$. Then $T$ acts on $X$ by
\[
T \times X \to X,\quad  (t,x) \mapsto t x t^{-1}.
\]

Then there exists a
unique element $\lambda\in X^*(T)$ such that $T$ acts on $X$ by
\[
T\times X\to^{\lambda,f} \Gm \times \Ga
\to \Ga \to^{f^{-1}} X,
\]
using the action of $\Gm$ on $\Ga$ given above.  The element
$\lambda\in X^*(T)$ attached to $X$ is called a positive root with
respect to $B$ and $T$, and $X = X_\lambda$ is called a positive root
space.  The set $\Phi^+$ of positive roots is a finite subset of
$X^*(T)$.

Let $\alpha\in\Phi^+$, let
$T_\alpha\subset T$ be the kernel of $\alpha: T\to \Gm$, and let
$M_\alpha$ be the centralizer of $T_\alpha$ with derived group
$M_\alpha'$.  The affine group $M_\alpha'$ is almost simple with
maximal torus $S \subset T \cap M_\alpha'$.  There exists a unique
cocharacter $\alpha^\vee\in X_*(T)$ factoring through $S$
\[
\alpha^\vee:\Gm\to S \to T
\]
and such that $\langle \alpha,\alpha^\vee\rangle = 2$.  Running over
$\alpha\in \Phi^+$, the set of all $\alpha^\vee$ so obtained is called the
set of positive coroots $\Phi^{\vee+}\subset X_*(T)$.


\subsection{simple roots and the Dynkin diagram}

There exists a unique subset $\Delta\subset \Phi^+$ such that every
positive root is a linear combination of $\Delta$ with nonnegative
coefficients.  That is, every positive root is in the cone spaned by
the simple positive roots.  The set $\Delta$ is called the set of
simple roots.  (The set $\Delta$ depends on the choices $T\subset
B\subset G$.)


\section{classification data}

In this subsection, we drop the earlier context.
Let $\Delta$ be any finite set.


Define the {\it Cartan matrix} to be a matrix $A:\Delta\times \Delta\to\ring{Z}$
that has the following properties.
\begin{itemize}
\item $A(\alpha,\beta)\in \{2,0,-1,-2,-3\}$.
\item $A(\alpha,\alpha)=2$.
\item $A(\alpha,\beta)\le 0$ for all $i\ne j$.
\item $A(\alpha,\beta)=0$ iff $A(\beta,\alpha)=0$.
\item If $A(\alpha,\beta)<-1$, then $A(\beta,\alpha)=-1$.
  \end{itemize}

The Dynkin diagram of $(\Delta,A)$ is a graph (with no loops and no
multiple joins) with $r$ nodes, indexed by $\Delta$, which we now
describe.  Edges of the graph are optionally decorated with an integer
$2$ or $3$ and an ordering of the endpoints (an arrow from node
$\alpha$ to node $\beta$).  Two distinct nodes are connected by an
edge iff $A(\alpha,\beta)\ne 0$.  The edge is not decorated if
$A(\alpha,\beta)=A(\beta,\alpha)=-1$.  But if $A(\beta,\alpha)\ne
-1,0$, then the edge is decorated with the integer $-A(\beta,\alpha)$
and an arrow from $\alpha$ to $\beta$.  The integer $2$ or $3$ is
often depicted as a double or triple bond in the graph.

An isomorphism of Dynkin diagrams is a bijection $f:\Delta\to\Delta$
that preserves edges and the labeling of on edges.

An arrow-forgetful automorphism $\rho$ of a Dynkin diagram is defined
to be a graph isomorphism of the underlying graph of the Dynkin diagram
except that it might not respect the optional arrows that label some edges.
$\rho$ is required to respect all the optional natural number labels.


A classification datum consists of a tuple
\[
(D,\rho,p,e)
\]
where $D$ is a Dynkin diagram, $\rho$ is an arrow-forgetful
automorphism of the diagram $p$ is a prime number, and $e$ is a
rational number.  Not all such tuples appear in the classification
theorem.  See definition
\label{def:st}.

In particular, $D$ must be only of the connected Dynkin diagrams that
appear in the Cartan classification of Lie algebras over $\ring{C}$.
By that classification, there are diagrams
\[
D = A_r,~ B_r,~ C_r,~ D_r,~ E_6,~ E_7,~ E_8,~ G_2,~ F_4.
\]
as shown in Figure \XX{insert}.

The automorphism $\rho$ of the Dynkin diagram is determined by its
order, and the symbol for the Dynkin diagram is decorated with a
prepended superscript indicating the order.  If $\rho=1$ the
superscript is omitted.  The tuple $(D,\rho,p,e)$ is thus written
${}^\rho D(q)$ or $D(q)$, where $q=p^e$.  Here is the classification.
$p$ is any prime. $e\in\ring{N}$, with $e\ge 1$.  The excluded values
come from \cite{Wiki}.

\begin{definition}\label{def:st}
  A tuple $(D,\rho,p,e)$ has {\it simple type}
  if it consists of one of the following cases.
\begin{itemize}
\item $A_r(p^e)$, where $r\ge 1$. If $r=1$, then $p^e>3$.
\item $B_r(p^e)$, where $r\ge 2$. If $r=2$, then $p^e > 2$.
\item $C_r(p^e)$, where $r\ge 3$.
\item $D_r(p^e)$, where $r\ge 4$
\item ${}^2A_r(p^e)$, where $r\ge 2$. If $r=2$, then $p^e > 2$.
\item ${}^2D_r(p^e)$, where $r\ge 4$.
\item ${}^3D_4(p^e)$.
\item $G_2(p^e)$, where $p^e > 2$.
\item $F_4(p^e)$.
\item $E_6(p^e)$.
\item ${}^2E_6(p^e)$.
\item $E_7(p^e)$.
\item $E_8(p^e)$.
\item ${}^2B_2(2^{f+1/2})$, where $f\in\ring{N}$.
\item ${}^2G_2(3^{f+1/2})$, where $f\in\ring{N}$.
\item ${}^2F_4(2^{f+1/2})$, where $f\in\ring{N}$.
\end{itemize}
\end{definition}

The last three cases are noteworthy.  If the automorphism $\rho$ of
$D$ is nontrivial, and if some edge of $D$ carries a label $a =2,3$,
then $p=a$.


\section{Frobenius}

Let $K$, $G$, $B$, $T$ be given as in the previous section.  We assume
in this section that $K$ is an algebraically closed field of positive
characteristic $p$.

Let $q = p^k$ for some $k\ge 1$.

There is a {\it Frobenius} morphism $F_q:GL(n)\to GL(n)$ given on
coordinate rings by
\[
F_q^*(x_{ij}) = x_{ij}^q,\quad F_q^*(x_0) = x_0^q.
\]
Let $G$ be a linear algebraic group over $K$.  {\it Standard
  Frobenius} data for $G$ consists of data $(F,n,\phi,k)$, where
$F:G\to G$ is a morphism, $\phi:G\to GL(n)$ is an isomorphism of $G$
onto a closed subgroup of $GL(n)$ that intertwines:
\[
\phi\circ F = F_q \circ \phi
\]
where $q= p^k$.  {\it Frobenius data} $(F,n,\phi,k,j)$ consists of a
morphism $F : G\to G$, where $(F^j,n,\phi,k)$ is a standard Frobenius.
Here $F^j = F\circ F\cdots$ ($j$ times).

Let $G$ be a reductive group with Frobenius data $(F,n,\phi,k,j)$.
Then $e = k/j\in\ring{Q}$ is the {\it Frobenius exponent}.

Assume further that $F$ has the property that $F(T)=T$ and $F(B)=B$.
The Frobenius $F$ then permutes the set of root spaces $X_\alpha$, of
simple roots $\alpha\in \Delta$ and there is a unique permutation
$\rho:\Delta\to\Delta$ such that $F(X_\alpha) = X_{\rho\alpha}$.  This
uniquely determines an automorphism (again called $\rho$) of the
Dynkin diagram that respects the integer labels on edges, but that
does not always respect the direction of the arrows on edges.

If $F$ if a Frobenius map on $G$, then the set of fixed points
\[
G(K)^F = \{ g \in G(K) \mid F(g) = g\}
\]
is a finite group. In fact,
\[
G(K)^F \subset^\phi GL(n,K)^{F^j} = GL(n,\ring{F}_{q}),
\]
where $\ring{F}_q = K^{Fr_q}$ is a finite field with $q=p^k$
elements.

We have a map $Q$ from the set of pairs $(G,F)$ to the set of finite
groups given by
\[
Q(G,F) = (G(K)^F/Z_G(K)^F)'
\]
Here the derived group (indicated by $'$) is the derived group as an
abstract group rather than the closed derived subgroup introduced
earlier.


\section{Almost Simple simply-connected groups}

We say that an almost simple group $G$ is {\it simply connected} if
the set of positive coroots spans the cocharacter lattice:
$\ring{Z}\Phi^{\vee+} = X_*(T)$.  (This condition does not depend on
the choice of $B$ and $T\subset B$.)

The structure theory of almost simple groups leads to a collection
of data $(p,K,G,B,T,\ldots)$ that we call the Lie {\it theater}.
This is the data introduced in the first pages of an enormous number
of research articles on Lie theory.  Here we consider the data
as restricted to the context of interest for the classification
of finite simple groups.

\begin{definition}  A {\it Lie theater} (for almost-simple simply connected
  groups) consists of the following data
  \[
  \Theta = (p,K,G,B,T,A,D,F,n,\phi,k,j,e,\rho)
  \]
  subject to the type constraints and propositions given below.
\end{definition}

\begin{itemize}
\item $p\ge 2$ is a prime number.
\item $K$ is an algebraically closed field of characteristic $p$.
\item $G$ is an almost simple group.
\item $B$ is a Borel subgroup of $G$.
\item $T$ is a maximal torus of $G$ that is also a subgroup of $B$.
  The maximal torus has character $X^*(T)$ and cocharacter groups $X_*(T)$.
\item Let $\Phi^+\subset X^*(T)$ be the set of positive roots in $X^*(T)$
  with respect to $B$ and $T$.
\item Let $\Delta\subset \Phi^+$ be the set of simple roots.
\item $A:\Delta\times\Delta\to\ring{Z}$ is the Cartan matrix
  $A(\alpha,\beta)=\langle \alpha,\beta^\vee\rangle$ with node set $\Delta\ni\alpha,\beta$.
\item $D$ is the Dynkin diagram attached to $\Delta,A$.
\item $(F,n,\phi,k,j)$ is Frobenius data for $G$ such that $F(G)=G$,
  $F(B)=B$ and $F(T)=T$.
\item $e = k/j\in \ring{Q}$ is the Frobenius exponent.
\item $\rho:D\to D$ is the arrow-forgetful automorphism of the Dynkin diagram induced
  by the Frobenius data (as described above).
\end{itemize}

The data is subject to the following additional conditions.

\begin{itemize}
\item (simply-connected)  $\ring{Z}\Phi^{\vee+} = X_*(T)$.
\item $(D,\rho,p,e)$ has simple type. (See definition \ref{def:st}.)
\end{itemize}

Given a theater $\Theta =(p,K,G,\ldots,F,\ldots)$ there is a finite
group $Q(\Theta) = Q(G,F)$, introduced above.

\begin{theorem}  If $\Theta$ is a Lie theater, then $Q(\Theta)$
  is a finite simple group.
\end{theorem}


\begin{definition}\label{def:fsg-Lie}
  A finite simple group isomorphic to $Q(\Theta)$ for some
  Lie theater $\Theta$ is a {\it finite simple group of Lie type}.
\end{definition}


We define a relation between tuples of simple type and finite simple
groups $Q$ of Lie type:
\[
(D,\rho,p,e)\sim H
\]
if there exists a theater $\Theta$ mapping to $(D,\rho,p,e)$ such that
$Q = Q(\Theta)$.

\begin{theorem}[classification of finite groups of Lie type]
  This relation $(\simeq)$ induces a well-defined surjective map
  from isomorphism classes of tuples of simple type
  to isomorphism classes of finite groups of Lie type.
  If two non-isomorphic tuples of simple type map to
  isomorphic finite groups of Lie type, then it is one
  of the duplicates enumerated in Theorem~\ref{thm:duplicate}.
\end{theorem}

That is, each tuple of simple type is isomorphic to one in the
image of a Lie theater $\Theta$.  Then $Q(\Theta)$ is a finite simple
group of Lie type.




\subsection{Notes on Lie type}

Wiki's list \cite{Wiki} notes that the order of a finite simple group
uniquely determines it up to isomorphism, with the following exceptions.
\begin{itemize}
\item $|\Alt_8|= |A_2(4)|$ and both have order 20160,
\item $|B_n(q)|=|C_n(q)|$, when $n>2$ and $q$ odd.
\end{itemize}

There is a special situation that occurs for the smallest of the
the series ${}^2B_2(2^{1/2})$.  It is not simple, but its derived
group is simple.  This is the Tits group, which is sometimes
considered as a sporadic groups.  It is for the
sole purpose of including the Tits group that the map $Q$
was defined using the derived subgroup.

\section{Comments about Notation}

The rest of the article deals with the sporadic groups.  We depart
from conventions used in the earlier sections, where $G$ was an affine
group.  Now $G$ is an abstract group; that is, a group in the
conventional sense.  Notation will be adapted to this new setting, so
that now $Z(G)$ and $G'$ denote the center and derived subgroup of
$G$, respectively.

$\Alt_n$ is the alternating group on $n$ letters,
(reserving $A_n$ for the Dynkin diagram that appeared earlier).

``$A \times B$ denotes a direct product, with normal subgroups $A$ and
$B$; also $A:B$ denotes a semidirect product (or split extension),
with a normal subgroup A and subgroup $B$; and $A \cdot B$ denotes a
non-split extension, with a normal subgroup $A$ and quotient $B$, but
no subgroup $B$; finally $A.B$ or just $AB$ denotes an unspecified
extension" \cite[p.9]{wilson2009finite}. (For sequences of dots, what
is the operator precedence and associativity of this notation?)

If $p$ is prime, $p^n$ denotes a direct product of $n$ cyclic groups
of order $p$.  That is, $p^n$ is elementary abelian group of order
$p^n$.  \cite[p.9]{wilson2009finite}


Let $p$ be a prime.  An extraspecial $p$-group is a finite $p$-group
$P$ such that $P' = Z(P)$ and $P'$ has order $p$.  Then $P$ has order
$p^{1+2n}$, for some positive $n$.  For each $p,n$, there are two
extraspecial $p$-groups of a given order up to isomorphism, denoted
$p^{1+2n}_{\pm}$. The signs are determined by the following rules.  If
$p$ is odd, the sign is $+$ iff $P$ has exponent $p$ (that is, every
non-identity element has order $p$).  If $p=2$, the function $x$ to
$x^2$ on $P$ induces a map $P/P'$ to $Z(P)$, which turns out to be a
quadratic form on the vector space $P/P'$.  The notation $+$ means
that the quadratic form has an isotropic vector, and $-$ otherwise
(for an anisotropic form) \cite[p.19]{robert1998twelve},
\cite[pp.59,83]{wilson2009finite}.

The term {\it almost simple} is used in the sense of Lie theory (that
is, having a simple Lie algebra, or equivalently noncommutative and
having no proper closed connected normal subgroups) and not in the
sense used by \cite[p.22]{wilson2009finite}.

$S_n$ is the symmetric group on $n$ letters, and $\Alt_n$ is the
alternating subgroup of index two in $S_n$.

$O_p(G)$ denotes the $p$-core of the finite group $G$.  It
is the largest normal $p$-subgroup of $G$.  It is the intersection
of Sylow $p$-subgroups.


\section{Sporadic Groups}

The lists above specify all but the last $26$ finite simple groups,
called the sporadic groups.  The $26$ sporadic groups are classified
into three happy families $(5+7+8)$ and the six pariahs.  A useful
reference is the atlas of sporadic groups \cite{A}.

\subsection{First Happy Family (Mathieu groups)}

The first happy family (of $5$) are the Mathieu groups.  These are quite
elementary to specify in terms of automorphism groups of Steiner
systems and their subgroups.

\begin{itemize}
\item A Steiner system $S(t,k,v)$, where $t < k < v$ are positive
  integers is a finite set $X$ of cardinality $v$, a collection of $k$
  element subsets of $X$ (called blocks), such that each $t$ element
  subset of $X$ is contained in a unique block.
\item The Steiner system $S(5,8,24)$ exists and is unique up to
  isomorphism.  The automorphism group of this Steiner system is
  $M_{24}$ (which by construction is a subgroup of $S_{24}$).  The
  stabilizer of a point $x$ (in the $24$ element set $X$) is $M_{23}$,
  and the stablizer of two distinct points is $M_{22}$.
\item The Steiner system $S(5,6,12)$ exists and is unique up to
  isomorphism.  The automorphism group of this Steiner system is
  $M_{12}$.  The stabilizer of a point is $M_{11}$.
\item The isomorphism class of the finite simple groups $M_{24}$,
  $M_{23}$, $M_{22}$, $M_{12}$, and $M_{11}$ does not depend on any of
  the choices in this construction.
\end{itemize}

\subsection{Second Happy Family and the Leech lattice}

The second happy family (of 7) are those related to the Leech lattice.
\begin{itemize}
\item The Leech lattice $L=L24$ can be characterized as the unique
  even unimodular lattice in dimension $24$ that does not have any
  vectors of norm 2.  Norm means the length squared here.
\item $Co_1 = Aut_0(L)/\{\pm 1\}$, where $A=Aut_0(L) \subset O(24)$ is
  the subgroup of the automorphism group of $L$ fixing the origin.
\item $Co_2 \subseteq A$ is the stabilizer of any vector of norm $4$
  (that is, any nonzero vector of shortest length).
\item $Co_3 \subseteq  A$ is the stabilizer of any vector of norm $6$.
\item $McL \subseteq Co_3$ is the stabilizer of any pair of vectors
  $v,w\in L$ such that $v$ has norm $6$, $w$ has norm $4$, and with
  inner product $(w,v) = -3$.  Equivalently $McL$ is the stabilizer of
  any pair of vectors $v,w$ of norm $4$, such that $(w,v) = -1$.
\item $HS \subseteq Co_3$ is the stabilizer of any pair of vectors
  $v,w$ in $L$ such that $v$ has norm $6$ and $w$ is a vector of norm
  $4$, such that $(w,v) = -2$.  Equivalently, $HS$ is a rank $3$
  permutation group on $100$ points in which a point stabilizer is
  isomorphic to $M_{22}$. \cite[p.116]{robert1998twelve}.  (Rank $3$
  permutation group is defined below in the pariah section.)
\item $J_2$.  $Co_1$ has a subgroup of order $3$ whose normalizer is
  isomorphic to $\Alt_9 \times S_3$.  The $\Alt_9$ factor has subgroup
  $\Alt_5$, and the centralizer of $\Alt_5$ in $Co_1$ is $J_2$
  \cite[p.218]{wilson2009finite}.
\item Suz.  In continuation of the construction of the subgroups
  $\Alt_9 \times S_3$ and $\Alt_5$ in $Co_1$, the group $\Alt_5$ has a
  subgroup $\Alt_3$.  The centralizer of $\Alt_3$ in $Co_1$ is $3\cdot
  Suz$, a $3$-fold central cover of $Suz$.
  \cite[p.218]{wilson2009finite}.
\item The isomorphism class of the finite simple groups $Co_1$,
  $Co_2$, $Co_3$, $McL$, $HS$, $J_2$, $Suz$ do not depend on any of
  the choices in the construction.
\end{itemize}

\subsection{monster}

The third happy family (of 8) are those related to the Fischer-Griess
monster group $M$.  A construction of the monster is needed.

\begin{itemize}
 \item
 Here is a construction of the monster (up to a factor of
 2) as the quotient of an infinite Coxeter group by one
 additional relation \cite{P}.  This gives a quick
 construction. (Compare \cite[sec.5.8.5]{wilson2009finite}.)
\item The monster can also be described up to isomorphism as the
  finite simple group that contains a pair of involutions $z,t$, such
  that the centralizer of $z$ is isomorphic to $2^{1+24}_{+}Co_1$ and
  the centralizer of $t$ is isomorphic to the baby monster
  $B$. \cite[p.148]{robert1998twelve}.  This approach requires us to
  have a prior construction of $B$.
\item A third possible route to the monster is the construction of the
  Griess algebra, which has the monster as its automorphism group.
  (See the sketches in \cite[p.146]{robert1998twelve} and
  \cite[p.251]{wilson2009finite}.)  A fourth construction is through
  the monster vertex algebra \cite{VOA}.
\end{itemize}

The monster group has an elementary presentation in terms of a
quotient of an infinite Coxeter group \cite{JNB} \cite{P}.

We define a Coxeter group $Y_{443}$.  The Coxeter graph is associated
with a Coxter graph with $12$ vertices:
\[
v_{ij},\quad i=0,1,2,\quad j=0,..,3.
\]
We construct an undirected graph with $9$ edges
$\{v_{ij},v_{i,{j+1}}\}$, for $j<3$, and with two additional edges
that connect the graph:
\[
\{v_{00},v_{10}\}, \{v_{00},v_{20}\}.
\]
Write $E$ for this set of $11$ edges.

Note: the graph is in the shape of a $Y$ with three arms $v_{ij}$, for
$i=0,1,2$.

The associated Coxeter group $Y_{433}$ is the group given by $12$
generators and the following relations.

Generators:
\[
x_{ij}, \quad i=0,1,2,\quad j=0,..,3.
\]

Relations:
\[
\forall i j, \quad (x_{ij}x_{i'j'})^{m(i,j,i',j')} = 1,
\]
where
\[
m(i,j,i',j') = 
\begin{cases}
1,&\text{if } (i,j) = (i',j');\\
3,&\text{if }  \{v_{ij},v_{i'j'}\}\in E;\\
2,&\text{otherwise}
\end{cases}
\]

This defines $Y_{433}$.  We define $G$ to be the quotient of $Y_{433}$ by
the additional relation (called the spider relation):
\[
(x_{00}x_{20}x_{21} x_{00}x_{10}x_{11} x_{00} x_{01} x_{02})^{10} = 1.
\]
This quotient is $2 \times M$.



\subsection{Third Happy Family and the Monster}

 
\begin{itemize}
\item $M$.  The monster is characterized above and has an atlas page
  with further useful information \cite{AM}.
\item $B$.  $2\cdot B$ is a centralizer of an involution in $2A$ in $M$
  \cite[p.256]{wilson2009finite}.  ($2A$ is specified below.)
\item $Fi_{24}'$.  $3\cdot Fi_{24}'$ is a centralizer of an element $x$ in
  the conjugacy class $3A$ of order $3$ in $M$ \cite[p.256-257]{wilson2009finite}.
  Moreover, $3\cdot Fi_{24}$ is the normalizer of the subgroup $\langle x\rangle $.
\item $Fi_{23}$.  This is a subgroup of $Fi_{24}'$ described below.
\item $Fi_{22}$.  This is a subgroup of $Fi_{24}'$ described below.
          %Equivalently, the normalizer $N_M(<6A>)$ of the subgroup
          %generated by an element in conjugacy class $6A$ of order
          %$6$ is $(S_3 \times 2\cdot Fi_{22}).2.$
\item $Th$.  Let $x$ be an element of order three in the conjugacy
  class $3C$, then $Th = C_M(\langle x\rangle )/\langle x\rangle $.
            %The monster has a involution whose centralizer is
          %isomorphic to $2^{1+24}\cdot Co_1$ \cite[p.253]{wilson2009finite}.  Take an
          %element ``$x$ of order three from $2^{1+24}\cdot Co_1$, such
          %that $x$ maps to an element of Co_1 with centralizer 3
          %\times \Alt_9."  Then N_M(\langle x\rangle ) \simeq S_3 \times Th
\item $HN$.  Let $x$ be an element of conjugacy class $5A$ in $M$.
  Then $HN \simeq C_M(x)/\langle x\rangle $ \cite[p.262]{wilson2009finite}.
  %\cite[p.260]{wilson2009finite}.  More directly,
          %``If x is an element of order 5 in 2^{1+24}\cdot Co_1,
          %mapping to an element of class 5B in the quotient group of
          %Co_1, then its centralizer has shape 5 \times 2^{1+8}\cdot
          %(\Alt_5 \times \Alt_5).2."  Then HN \simeq C_M(x)/\langle x\rangle 
          %\cite[p.262]{wilson2009finite}.  (5B in Co_1 is \cite{ACo1} defined in the
          %altas.) More directly,
\item He.  Let $x$ be an element in conjugacy class $7A$ in $M$.  Then
  $He \simeq C_M(x)/\langle x\rangle.$ \cite[p.263]{wilson2009finite}.  %More directly,
          %If x is any element 7A in M, then its normalizer in
          %M is (7:3 \times  He):2, where He is the Held group
\end{itemize}

These constructions mention conjugacy classes $2A$, $3A$, $3C$, $5A$,
$6A$, $7A$ in $M$. Classes are labeled $NX$, where $N=1,2,3,4,5,6$ is
the order of an element in the conjugacy class, and $X=A,B,\ldots$ are
letters to differentiate the conjugacy classes of a given order, $A$
is the class of smallest cardinality of that order, $B$ is the second
smallest, etc.  In general there can be classes of the same order and
cardinality, but that does not happen with any of the orders in $M$ we
discuss.  In particular, $2A$ is the smallest conjugacy class of order
$2$, and $3A$ is the smallest conjugacy class of order $3$.  The
product of two elements of $2A$ can land in any of nine different
conjugacy classes: $1A$, $2A$, $2B$, $3A$, $3C$, $4A$, $4B$, $5A$,
$6A$.  See \cite[p.7]{Ga} and \cite[p256]{wilson2009finite}.

A $3$-transposition group is a ``finite group $G$ generated by a
conjugacy class $D$ of involutions, such that any two elements of $D$
have a product of order at most $3$, and ... such that $G' = G''$ and
any normal $2$- or $3$-subgroup of $G$ is central''
\cite[p235]{wilson2009finite}.  We call the elements of $D$
transpositions.  There is a classification of $3$-transposition
groups.  There is a $3$-transposition group such that its quotient by
its center is $Fi_n$, for $n=22,23,24$.

\begin{itemize}
\item $Fi_{23}$.
  The stabilizer of one transposition in $Fi_{24}'$ is $Fi_{23}$.
\item $Fi_{22}$.  The stabilizer of two commuting transpositions in
  $Fi_{24}'$ is $2\cdot Fi_{22}:2$. \cite[p.250]{wilson2009finite}
\end{itemize}

\subsection{The Pariahs}

It is not clear what the best route to the six pariahs will be.
Expert opinion is needed.  At the very least we might specify them as
the unique simple group of a given order, even if that is not a very
useful description.  ``The behavior of these six groups is so bizarre
that any attempt to describe them ends up looking like a disconnected
sequence of related facts." \cite[p.184]{wilson2009finite}.
\cite[pp.150-153]{robert1998twelve} gives a brief description of each
pariah.

\begin{itemize}
\item $J_1$.  ``It is the unique simple group which has abelian Sylow
  $2$-groups and contains an involution with centralizer isomorphic to''
  $2 \times \Alt_5$.   \cite[p.150]{robert1998twelve}
          %Moreover, it was
          %"constructed by computer by John Cannon as a subgroup of
          %G2(11) generated by a pair of degree 7 matrices over the
          %integers modulo 11."
\item $J_3$. This group appears in a centralizer involution problem.
  %``This is a group discovered by Janko as a solution to a
  %centralizer of involution problem.  ``The involution centralizer is
  $2^{1+4}_{-}:\Alt_5$, split extension (therefore not the standard
  holomorph, which embeds in $GL(4,C)$)''
  \cite[p.151]{robert1998twelve}.  There are two finite simple groups
  with this involution centralizer.  The other is $J_2$, which appears
  in the second happy family.  $J_3$ has order $2^7\,3^5\,5.17.19$."
\item $Ly$.  This group is uniquely characterized up to isomorphism as
  a finite simple group that contains an involution whose centralizer
  is isomorphic to $2 \cdot \Alt_{11}$, the covering group of the
  alternating group of degree $11$.  Moreover, the group has a single
  class of involutions and has order
  $2^8*3^5*5*7*11*37*61*67$. \cite[p.151]{robert1998twelve}
\item $O'N$. This is the unique simple group $G$ up to isomorphism
  with the following properties \cite[p.152]{robert1998twelve}:
\begin{itemize}
\item For every elementary subgroup $E$ of $G$, we have
 $N_G(E)/C_G(E) \simeq GL(E)$.
 %[correcting a typo: $N_E(E)$ to N_G(E).]
 \item $G$ contains a subgroup $E \simeq 2^3$
 such that $N_G(E) \simeq 4^3 \cdot 
 GL(3,2)$
 \item $G$ contains an involution whose centralizer has the form
 $4.PSL(3,2).2$.
\end{itemize}
%Moreover, ``this group has an outer automorphism of order $2$ whose fixed
%point subgroup is isomorphic to $J_1$."
\item $J_4$. This is characterized as a finite simple group that
  contains an involution with centralizer isomorphic to
  $2^{1+12}_{-}\cdot 3\cdot M_{22}:2$.  ``The centralizer is nonsplit
  over the [$2$-core] $O_2$ and contains a perfect group $6\cdot
  M_{22}$."  Moreover, the simple group has order
  $2^{21}3^35.7.11^3.23.29.31.37.43$ \cite[p.152]{robert1998twelve}.
\item Ru.  A permutation group is defined to have rank $r$ if it is
  transitive and a point stabilizer has $r$ orbits.  This group is
  characterized as a rank $3$ finite simple group whose point
  stabilizer is $H = {}^2F_4(2)$, the Ree group over the field with
  $2$ elements. (As described above, the derived group of $H$ is the
  Tits group.) \cite[p.151]{robert1998twelve}
\end{itemize}
 

\section{Checking the Answer}

If proofs are not included in the statement of the theorem, what
sanity checks are there to make sure that the specification is
correct?  One possibility that has been discussed is the following.
There is currently a bridge between Mathematica and Lean, constructed
by Rob Lewis and Minchao Wu \cite{L}.  Lean is also now available as a
package in CoCalc (which also provides access to GAP).  We can hope
that further development will provide a bridge (at some indefinite day
in the future) that would allow for some sanity checks by running some
GAP calculations based on the Lean specification and comparing to
known answers.  We can discuss what might be done here.


Other references \cite{aschbacher1994sporadic}, \cite{carter1985finite}.




